\chapter{Plasma coronal}

\section{Emisividad EUV}
\textcolor{red}{Seccion 2.2 de la tesis de lic.}
\subsection{Instrumentos EUV}
\textcolor{red}{
\begin{itemize}
  \item EIT/soho
  \item EUVI/stereo
  \item AIA/sdo
\end{itemize}
}
\section{Emisividad en luz blanca}
\textcolor{red}{buscar el paper!}
\subsection{Instrumentos LB utilizados}
\textcolor{red}{
\begin{itemize}
  \item LASCO-C2/soho
  \item K-Cor/HAO
\end{itemize}
}

\section{MHD}
\textcolor{red}{Introducir MHD como descripcion principal en el plasma coronal. Comentar todas las aproximacions, de donde se parte y a donde se llega. Definir Reynold magnético y de la ec de induccion explicar que para Rm grande el campo magnetico es arrastrado por el fluido. Notar que esto explica buena parte de la estructura del viento solar, IMF, HCS, etc.
\begin{itemize}
  \item MHD ideal
  \item Frozen-in
  \item Rm
\end{itemize}
}

\section{Modelo AWSoM}
Damping of Alfvén wave turbulence as a source of coronal heating has also been extensively studied from the early days of in situ solar wind observations (e.g., Barnes 1966, 1968). Later, it was demonstrated that reflection from sharp pressure gradients in the solar wind (Heinemann and Olbert 1980; Leroy 1980) is a critical component of Alfvén wave turbulence damping (Matthaeus et al. 1999; Dmitruk et al. 2002; Verdini and Velli 2007). For this reason, many numerical models explore the generation of reflected counter-propagating waves as the underlying cause of the turbulence energy cascade (e.g., Cranmer and Van Ballegooijen 2010), which transports the energy of turbulence from the large-scale motions across the inertial range of the turbulence spatial scale to short-wavelength perturbations. The latter can be efficiently damped due to wave–particle interaction. In this way, the turbulence energy is converted to random (thermal) energy.
\textcolor{red}{
\begin{itemize}
  \item Importancia de un modelo MHD 3D
  \item Modelo AWSoM
\end{itemize}
}





%\chapter{Resultados}
%\section{Paper1}
%\section{Paper2}
%\section{Paper3}
