%% This is an example first chapter.  You should put chapter/appendix that you
%% write into a separate file, and add a line \include{yourfilename} to
%% main.tex, where `yourfilename.tex' is the name of the chapter/appendix file.
%% You can process specific files by typing their names in at the 
%% \files=
%% prompt when you run the file main.tex through LaTeX.
%\chapter{Introduction}

\chapter{El Sol y la corona solar}

\section{Interior Solar}
\textcolor{red}{
Esta sección introduce el interior solar, de que esta conformado, como produce energia y como se transmite hacia el exterior.
\begin{itemize}
  \item nucleo (fusión)
  \item zona radiativa y convectiva
  \item campo magnetico
  \item manchas solares
\end{itemize}
}
\section{Atmósfera solar}
\textcolor{red}{En esta sección se introduce el objeto de estudio principal de la tesis, no escatimar en detalles. 
\begin{itemize}
  \item Fotosfera
  \item cromosfera
  \item region de transicion (referencia al problema del calentamiento coronal)
  \item extension hasta el medio interplanetario
  \item eventos importantes ocurren a baja altura y alta altura
\end{itemize}
}
\subsection{La corona solar}
%\section{El problema del calentamiento coronal}
The Sun’s visible surface, called the photosphere, is a layer of about 6000 K. The photosphere emits the so-called visible light (380–750 nm), peaking at a green wavelength of 500 nm. Averaged over its visible disk, the Sun’s emission is approximated by the black-body radiation of 5770 K. Above it is a thin layer of about 10000 K called the chromosphere; its average thickness is about 2000 km, much smaller than the solar radius of 7 .10 5 km. Its name (meaning ‘color sphere’) comes from its pinkish color surrounding a dark Sun in total eclipses, due to the H, (hydrogen Balmer-,) line of 656.3 nm. Also seen in total solar eclipses is the solar corona (Fig. \red{foto eclipse total san juan}), a pearly halo extending further out to a few solar radii. The solar corona is a tenuous outer atmosphere of the Sun, with a temperature of 1–2 MK. Although the existence of the corona has long been known by total solar eclipses, its million-degree temperature was not recognized until its spectra were correctly interpreted by the physics of radiation processes in the 1940s.
Then the question was raised why such a high temperature of the corona is brought about. Since the density of the corona (10 8–9 cm !3 ) is much smaller than that of the photosphere (10 17 cm !3 ), the thermal energy density of the corona (although it is 200–300 times hotter than the photosphere) is negligibly small compared with the photospheric energy density. However, due to the second law of thermodynamics, heat cannot flow from the photosphere to the corona to raise the coronal temperatures hotter than the photosphere. (Conversely, the heat actually flows from the corona to the photosphere; see Fig. 8.) Since there is no plausible source of energy further out in the corona to heat up the corona, we must assume that some form of energy other than heat is supplied from below the corona to realize its million-degree temperature. This is the coronal heating problem discussed in this article.
Figure \red{Figura 1.5 tesis Cecere} shows the temperature and density structures of the solar atmosphere. A thin boundary between the chromosphere and the corona is called the transition region, where the temperature suddenly jumps from 10 4 K to 10 6 K. The transition region emits ultraviolet (UV) and extreme ultraviolet (EUV) radiations. The emission from the corona of 1–2 MK is in the EUV to soft X-ray ranges. The visible light seen in total solar eclipses is photospheric light scattered by free electrons in the corona, and not ‘emitted’ by coronal plasmas. The emission lines seen in the visible wavelengths during total eclipses are due to a peculiar process described below, and should not be regarded as the main radiation from coronal plasmas. \citet{sakurai_2017}

\subsection{Viento solar}
\textcolor{red}{
Que es? que importancia tiene (impacto en la magnetosfera terrestre)? Comentar sobre el clima y la prediccion espacial. comentar sobre la formacion, morfologia, espiral de parker, aceleracion.
\begin{itemize}
  \item espiral de parker
  \item importancia en la prediccion del clima espacial
  \item propiedades
  \item Ulysses (mensionar grafico y paper)
  \item IMF y HCS (se puede evidenciar esto con graficos del modelo MHD ENLIL de ser necesario)
  \item viento solar arrastra el campo magnetico
\end{itemize}
}

\section{El ciclo de actividad solar}
\textcolor{red}{
Debe explicarse principalmente la descripcion fenomenologica de las observaciones que la caracteriza. En el paper llamado Solar Cycle del 2015 de Hathaway esta todo detallado.
\begin{itemize}
  \item Manchas solares y flujo F10.7 como indicadores
  \item diagrama mariposa
  \item grafico Hathaway
\end{itemize}
}


\begin{comment}
\chapter{Modelo MHD}

\textcolor{red}{
\begin{itemize}
  \item Importancia de un modelo MHD 3D
  \item Modelo AWSoM
\end{itemize}
}
\section{Modelo AWSoM}
Damping of Alfvén wave turbulence as a source of coronal heating has also been extensively studied from the early days of in situ solar wind observations (e.g., Barnes 1966, 1968). Later, it was demonstrated that reflection from sharp pressure gradients in the solar wind (Heinemann and Olbert 1980; Leroy 1980) is a critical component of Alfvén wave turbulence damping (Matthaeus et al. 1999; Dmitruk et al. 2002; Verdini and Velli 2007). For this reason, many numerical models explore the generation of reflected counter-propagating waves as the underlying cause of the turbulence energy cascade (e.g., Cranmer and Van Ballegooijen 2010), which transports the energy of turbulence from the large-scale motions across the inertial range of the turbulence spatial scale to short-wavelength perturbations. The latter can be efficiently damped due to wave–particle interaction. In this way, the turbulence energy is converted to random (thermal) energy.
\section{AWSoM}\label{ch1:opts}
\end{comment}




% This is an example of how you would use tgrind to include an example
% of source code; it is commented out in this template since the code
% example file does not exist.  To use it, you need to remove the '%' on the
% beginning of the line, and insert your own information in the call.
%
%\tagrind[htbp]{code/pmn.s.tex}{Post Multiply Normalization}{opt:pmn}


% This is an example of how you would use tgrind to include an example
% of source code; it is commented out in this template since the code
% example file does not exist.  To use it, you need to remove the '%' on the
% beginning of the line, and insert your own information in the call.
%
%\tgrind[htbp]{code/be.s.tex}{Block Exponent}{opt:be}
