% $Log: abstract.tex,v $
% Revision 1.1  93/05/14  14:56:25  starflt
% Initial revision
% 
% Revision 1.1  90/05/04  10:41:01  lwvanels
% Initial revision
% 
%
%% The text of your abstract and nothing else (other than comments) goes here.
%% It will be single-spaced and the rest of the text that is supposed to go on
%% the abstract page will be generated by the abstractpage environment.  This
%% file should be \input (not \include 'd) from cover.tex.
\chapter*{Resumen}
\addcontentsline{toc}{chapter}{Resumen}

\begin{center}
%%Debido al mecanismo de dínamo del Sol, la actividad magnética de su atmósfera presenta un ciclo periódico de unos 11 años de duración. Los últimos tres ciclos parecen indicar una disminución sistemática de la actividad solar. En particular el último mínimo (2009) presentó características anómalas, habiendo sido considerablemente más calmo que el mínimo previo (1996), así como magnéticamente más débil. Presentamos un estudio observacional comparativo de la estructura termodinámica tri-dimensional (3D) de la baja corona durante los dos últimos mínimos solares basado en la técnica de \emph{tomografía de medida de emisión diferencial} (DEMT, por sus siglas en inglés). El estudio es complementado con extrapolaciones potenciales del campo magnético coronal. Presentamos además un análisis de incertezas sistemáticas y finalizamos con un análisis de consistencia entre resultados DEMT con un modelo hidrostático isotérmico. Para estudiar los mínimos de los años 1996 y 2009 utilizamos datos de los instrumentos EIT/SoHO y EUVI/STEREO, respectivamente. El objetivo del trabajo es establecer si existieron { diferencias sistemáticas} en el estado termodinámico global de la baja corona durante ambos mínimos.

Se propone que el becario desarrolle estudios observacionales de la corona y el viento solar para rotaciones con diverso nivel de actividad, seleccionadas duarante los dos últimos ciclos de actividad solar. Simultáneamente el becario realizará simulaciones magnetohidrodinámicas (MHD) de los períodos estudiados. Desde el punto de vista observacional, se utlizará tomografía de medida de emisión diferencial (DEMT, por sus siglas en inglés) y tomografía solar rotacional (SRT, por sus siglas en inglés) en luz blanca para reconstruir la esturctura tridimensional (3D) de la densidad y temperatura electrónicas en rangos de alturas complementarios. Asimismo se planea utilizar observaciones espectales en el rango UV para obtener diagnósticos coronales complementarios, y mediciones in-situ del viento solar en la heliosfera extendida. Desde el punto de vista teórico, se realizarán simulaciones MHD de los períodos estudiados mediante el modelo Space Weather Modeling Framework(SWMF), utilizando en particular sus módulos cromosférico, coronal, y heliosférico. Se prevee poner especial énfasis en el estudio de las estructuras de streamers y pseudo-streamers, asociadas a la componente lenta del viento solar. El proyecto propuesto aportará resultados que permitirán profundizar la comprensión de los mecanismos físicos resonsables del calentamiento coronal y de la génesis del vientosolar lento, temas de investigación abiertos de la física solar.


A continuación de la Carátula, debe figurar un resumen del trabajo, junto con palabras claves asociadas. Luego, en página aparte, el título de la Tesis, el resumen y las palabras claves, traducido al inglés.


*Resumen en español + palabras claves \\
*Titulo de la tesis\\
*Resumen en ingles +palabras claves\\
*OBS: El resumen debe tener algo de motivacion y ademas incluir que se va a mostrar en cada capitulo.\\


\end{center}

\newpage~
\newpage
\pagenumbering{arabic} 


